%%%%%%%%%%%%%%%%%%%%%%%%%%%%%%%%%%%%%%%%%%%%%%%%%%%%%%%%%%%%%%%%%%%%%%%%%%%%%
\documentclass[12pt]{article}

\usepackage{edXpsl} % edX "problem specification language"

\begin{document}

% edXcourse: {course_number}{course display_name}[optional arguments like semester]
\begin{edXcourse}{MIT.latex2edx}{latex2edx demo course}[semester="2014 Spring"]

% edXchapter: {chapter display_name}[optional arguments like url_name]
\begin{edXchapter}{Basic examples}

% edXsection: {section display_name}[optional arguments like url_name]
% this turns into a <sequential> in the XML
\begin{edXsection}{Basic example problems}

% edXvertical: {vertical display_name}[optional arguments like url_name]
\begin{edXvertical}

% edXproblem: {problem display_name}{attributes: url_name, weight, attempts}
\begin{edXproblem}{Text Area}{attempts=10}

What is the numerical value of $\pi$?

% \edXabox: answer box, specifying question type and expected response
\edXabox{expect="3.14159" type="custom" tolerance='0.01' rows='10' cols='10' inline='1' cfn=''}

\end{edXproblem}
\end{edXvertical}
\end{edXsection}
\end{edXchapter}
\end{edXcourse}
\end{document}

%%%%%%%%%%%%%%%%%%%%%%%%%%%%%%%%%%%%%%%%%%%%%%%%%%%%%%%%%%%%%%%%%%%%%%%%%%%%%